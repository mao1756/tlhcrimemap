\documentclass{beamer}

\title{Tallahassee Crime Map}
\author{Munawar Ali, Çağatay Ayhan, Ece Karaçam, Abdullah Malik, and Mao Nishino}

\date{\today}

\begin{document}

\begin{frame}
    \titlepage
\end{frame}

\begin{frame}
    \frametitle{Outline}
    \tableofcontents
\end{frame}

\section{Introduction}
\begin{frame}
    \frametitle{Introduction}
    Write two goals of the project: 1) to create a crime map of Tallahassee, and 2) to analyze the crime data using traditional machine learning techniques.
\end{frame}

\section{Data Collection and Processing}
\begin{frame}
    \frametitle{TOPS Data Collection}
    Write a brief description about how we collected the data from TOPS
\end{frame}

\begin{frame}
    \frametitle{Data Processing}
    Write a short description about how we create our map dataset
\end{frame}

\section{Experimental Data Analysis}
\begin{frame}
    \frametitle{Temporal Analysis}
    Put figures about temporal trends in crime data (per semester, per month, per day, per hour, etc.)
\end{frame}

\begin{frame}
    \frametitle{Spatial Analysis}
    Put the figure about locations where crimes are frequent
\end{frame}

\section{Crime Heatmap Generation}

\begin{frame}
    \frametitle{pix2pix}
    Brief explanation of GAN and pix2pix
\end{frame}

\begin{frame}
    \frametitle{Results}
    Put the results of the pix2pix model, i.e., the generated heatmap and accuracy
\end{frame}

\begin{frame}
    \frametitle{Improving the city via generated heatmap}
    Edit a geographical map and reduce the crime rate predicted by the model
    Explain how it can be used by city planners
\end{frame}

\section{Conclusion}
\begin{frame}
    \frametitle{Conclusion}
    Summarize the results and a bit of future work (like a web app for the model)
\end{frame}

\end{document}