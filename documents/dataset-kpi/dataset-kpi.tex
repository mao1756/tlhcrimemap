\documentclass[10pt]{article}
\usepackage[margin=20truemm]{geometry}
\usepackage{xcolor}
\geometry{a4paper}

\usepackage{graphicx}

\title{Project Proposal: Tallahassee Crime Map}
\author{Munawar Ali, Çağatay Ayhan, Ece Karaçam, Abdullah Malik, and Mao Nishino}
\date{\today}

\begin{document}
\maketitle

\section{Project Summary}
In this project, we aim to analyze geographical crime data in Tallahassee, FL, collected from Tallahassee Police Statistics (TOPS). To clean up the data, we will conduct statistical and spatial analyses to identify crime patterns and trends. A key feature of our approach is the utilization of the pix2pix generative model, which allows for the prediction of crime heat maps from both existing and hypothetical geographical maps from this cleaned data. This enables urban planners to input their future planning designs into our model to identify potential crime hotspots in advance. Additionally, we plan to develop a web application to visualize the data, providing accessible insights to the public.

\section{Objectives}
\begin{itemize}
    \item We aim to analyze crime data to identify patterns and trends around neighborhoods.
    \item Create a heatmap of crime in Tallahassee using \verb|pix2pix| from a given geographical layout.
    \item Ultimately, we hope to develop a web application to visualize the data to help urban planners in designing their layouts.
\end{itemize}

\section{Dataset}
The dataset is collected from Tallahassee Police Statistics, which includes geographical information of police reports in Tallahassee. The dataset includes the following columns:
\begin{itemize}
    \item The time and date of the crime
    \item The location of the crime
    \item Report number
    \item The type of the report
    \item The geographical coordinates of the crime
    \item Location name of the crime, if given.
\end{itemize}

\section{Stakeholders}
The following are the stakeholders of the project:
\begin{itemize}
    \item Tallahassee Police Department
    \item Tallahassee residents
    \item Local businesses
    \item Urban planners
\end{itemize}


\section{KPIs}
We can quantify the success of our project based on the following metrics:
\begin{itemize}
    \item Crime rate by location as predicted by the heatmap.
    \item Accuracy of the heatmap.
\end{itemize}

\end{document}
